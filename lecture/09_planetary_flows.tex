% !TEX root = ./physics_of_fluids.tex
% !TEX TS-program = xelatex
% !TEX encoding = UTF-8 Unicode
\chapter[Planetary flows and the effect of rotation]{{\bfseries Planetary} flows and the effect of {\bfseries rotation}} \label{chap:planetary_flows}

Flows at the Earth scale are characterized by considerable horizontal
lengthscales (scaling with the radius of the Earth $\sim$ 6000 km) and
relatively small vertical lengthscales -- atmosphere is typically 10 km-thick
and oceans 4 km-deep. The ratio between these lengthscales is $O(1000)$ (which
remarkably is similar to the ratio between the radius of a soap bubble and the
thickness of the liquid film!). This contrast between lengthscales implies that
flows at the Earth-scale will be well captured by thin-film equations.

Another key aspect of planetary flows is \textbf{rotation}, which gives rise to
particular dynamics, of which cyclones are probably the most impressive
illustration. Cyclones owe their existence to the Coriolis force, which
imprints its mark on motion over large scales on Earth.

\section{Rotating fluids}
\label{sec:rotating_fluids}
The equations for motion in a rotating frame are best obtained using the Lagrangian formalism. At the fluid particle level, Newton's equation for motion is written:
\begin{equation}
  \label{eq:newton}
  \rho \dd{^2\br}{t^2} = \bff
\end{equation}
in an inertial frame. This equation can be rewritten in a rotating frame as:
\begin{equation}
  \label{eq:newton_rot}
  \rho \lp\dd{^2\br}{t^2} + 2 \bOmega \times \dd{\br}{t} + \bOmega \times \lp\bOmega \times \br\rp\rp = \bff,
\end{equation}
or:
\begin{equation}
  \label{eq:newton_rot_euler}
  \rho \lp\dd{\bu}{t} + 2 \bOmega \times \bu + \bOmega \times \lp\bOmega \times \br\rp\rp = \bff.
\end{equation}
Expressing the forces, we obtain:
\begin{equation}
  \label{eq:newton_rot_euler_forces}
  \rho \lp\dd{\bu}{t} + 2 \bOmega \times \bu + \bOmega \times \lp\bOmega \times \br\rp\rp = -\nabla p + \mu \Delta \bu + \rho \bg_\text{grav}.
\end{equation}
The centrifuge force part can be absorbed in a potential called the \textbf{geopotential} $\phi$:
\begin{equation}
  \label{eq:geopotential}
  \phi = g_\text{grav}z + \frac{1}{2} \lp\bOmega \times \br\rp^2.
\end{equation}
Introducing the effective gravity $\bg_\text{eff} = -\nabla \phi = \bg_\text{grav} - \bOmega \times \lp\bOmega \times \br\rp$, we can rewrite the equation of motion as:
\begin{equation}
  \label{eq:newton_rot_euler_forces_eff}
  \rho \lp\dd{\bu}{t} + 2 \bOmega \times \bu\rp = -\nabla p + \mu \Delta \bu - \rho \nabla \phi.
\end{equation}

We can non-dimensionalize this equation:
\begin{equation}
  \label{eq:newton_rot_euler_forces_eff_nondim}
  \frac{\rho U}{T} \pd{\bar{\bu}}{\bar t} + \frac{\rho U^2}{L} \lp \bar{\bu} \cdot \bar \nabla\rp \bar{\bu} + 2 \Omega \rho U \be_z \times \bar{\bu} = -\frac{P}{L} \bar\nabla \bar p + \frac{\mu U}{L^2} \bar\Delta \bar{\bu} - \frac{\rho \Phi}{L} \bar\nabla \bar\phi.
\end{equation}
In this expression, we will take for the natural time scale of the problem $T = \lp 2 \Omega\rp^{-1}$ which is the Coriolis pulsation. Further posing $P = \rho U \lp 2 \Omega \rp L$ and $\Phi = 2 \Omega U$ to keep the corresponding terms in the equation, we obtain:
\begin{equation}
  \label{eq:newton_rot_euler_forces_eff_nondim2}
  \pd{\bar{\bu}}{\bar t} + \text{Ro} \lp \bar{\bu} \cdot \bar \nabla\rp \bar{\bu} + \be_z \times \bar{\bu} = -\bar\nabla \bar p + E \bar\Delta \bar{\bu} - \bar\nabla \bar\phi.
\end{equation}
Two non-dimensional parameters appear in this equation:
\begin{itemize}
  \item the \textbf{Rossby number} $\text{Ro} = \frac{U}{2 \Omega L}$, which quantifies
        the relative importance of inertial terms with respect to the Coriolis force,
        and
  \item the \textbf{Ekman number} $E = \frac{\mu}{2 \rho \Omega L^2}$, which measures
        viscous friction relative to Coriolis action.
\end{itemize}
With a rotation rate of $\Omega = 7.3 \times 10^{-5} \text{s}^{-1}$,
we see that atmospheric winds of $\sim$100 km/h (28 m/s) will be dominated by rotation if $L \gtrsim 200$ km.
Oceanic flows are much slower; the Gulf Stream, which flows at $U \sim 1 \text{ m/s}$ sees this scale fall down to $7$ km. For the same oceanic flows, the Ekman number is $E \sim 10^{-10}$, showing that viscous friction is negligible compared to Coriolis action.

Flows at the planetary scale are therefore strongly influenced by rotation, and
to first approximation these flows result from a balance between pressure
gradient and Coriolis force. This is the \textbf{geostrophic balance}.

\section{Fluid flows on a rotating sphere}
\label{sec:flows_rotating_sphere}
The flows developing on the planet are complex but, depending on their scales,
simplifications may be made. Let's investigate these different regimes,
starting by writing the equations of motion on a rotating sphere.

\subsection{Governing equations in spherical coordinates}
We have for mass conservation:
\begin{equation}
  \label{eq:mass conservation_sphere}
  \frac{\mathrm d\rho}{\mathrm dt} + \rho \left(\pd{w}{r} + \frac{2w}{r} + \frac{1}{r \cos \theta} \pd{\lp v \cos \theta \rp}{\theta} + \frac{1}{r \cos \theta} \pd{u}{\phi}\right) = 0,
\end{equation}
where $(u, v, w)$ are the velocity components in the $(\phi, \theta, r)$
directions respectively (i.e. longitude, latitude, radius). $\dd{}{t}$ stands for the particle derivative operator\index{derivative!particle}, which reads in spherical coordinates:
\begin{equation}
  \label{eq:particle_derivative_spherical}
  \dd{}{t} = \pd{}{t} + \frac{u}{r \cos \theta} \pd{}{\phi} + \frac{v}{r} \pd{}{\theta} + w \pd{}{r}.
\end{equation}
The momentum equations take the following form:
\begin{subequations}
  \begin{empheq}[left=\empheqlbrace]{alignat=2}
    \rho \lp \dd{u}{t} + \frac{uw}{r} - \frac{uv}{r} \tan \theta - 2 \Omega \sin \theta v + 2 \Omega \cos \theta w\rp \,&=\ -\frac{1}{r \cos \theta} \pd{p}{\phi} + \matF_\phi,\\
    \rho \lp \dd{v}{t} + \frac{vw}{r} + \frac{u^2}{r} \tan \theta + 2 \Omega \sin \theta u \rp \,&=\ -\frac{1}{r} \pd{p}{\theta} + \matF_\theta,\\
    \rho \lp \dd{w}{t} - \frac{u^2+v^2}{r} - 2 \Omega \cos \theta u \rp \,&=\ -\pd{p}{r} - \rho g + \matF_r.
  \end{empheq}
\end{subequations}
Finally the energy equation reads:
\begin{equation}
  \label{eq:energy_sphere}
  \rho c_p \dd{T}{t} = k \Delta T + Q.
\end{equation}
Here $Q$ represents a possible volumetric heating (e.g. radiative heating from the Sun).

Not all the terms in these equations are equally important. In order to
identify the leading terms in the previous equations we \textbf{choose}
relevant scales for the problem and non-dimensionalize the equations.

\subsection{A local cartesian coordinate system}
\label{sec:local_cartesian}
In order to simplify the previous system of equations, we define the following variables:
\begin{subequations}
  \begin{empheq}[left=\empheqlbrace]{alignat=2}
    x \,&=\ R \phi \cos \theta_0,\\
    y \,&=\ R (\theta - \theta_0),\\
    z \,&=\ r - R,
  \end{empheq}
\end{subequations}
and relate the derivatives in the two coordinate systems:
\begin{subequations}
  \begin{empheq}[left=\empheqlbrace]{alignat=2}
    \pd{}{\phi} \,&=\ R \cos \theta_0 \pd{}{x},\\
    \pd{}{\theta} \,&=\ R \pd{}{y},\\
    \pd{}{r} \,&=\ \pd{}{z}.
  \end{empheq}
\end{subequations}
\subsection{Scaling the equations}
We now introduce characteristic scales and non-dimensional variables:
\begin{subequations}
  \begin{empheq}[left=\empheqlbrace]{alignat=2}
    x, y \,&=\ L \bar{x}, L \bar{y},\\
    z \,&=\ H \bar{z},\\
    u, v \,&=\ U \bar{u}, U \bar{v},\\
    w \,&=\ \epsilon \bar{w},\\
    t \,&=\ \frac{L}{U} \bar{t},
  \end{empheq}
\end{subequations}
with $\epsilon = H/L \ll 1$ the aspect ratio of the flow.

In order to scale pressure and density, we will consider the the
atmospheric/oceanic flows to be fluctuations around the hydrostatic equilibrium
given by
\begin{equation}
  \pd{p_0}{z} = -\rho_0 g,
\end{equation}
i.e. we write:
\begin{subequations}
  \begin{empheq}[left=\empheqlbrace]{alignat=2}
    p \,&=\ p_0(z) + \tilde{p},\\
    \rho \,&=\ \rho_0(z) + \tilde{\rho}.
  \end{empheq}
\end{subequations}
If we are located mid-latitude, we expect the pressure to be balanced by the Coriolis acceleration:
\begin{equation}
  \rho_0 2 \Omega \sin \theta_0 U \sim \frac{\tilde p}{L}.
\end{equation}
Noting $f_0 = 2 \Omega \sin \theta_0$ the Coriolis parameter at latitude $\theta_0$, we can therefore scale pressure fluctuations as:
\begin{equation}
  \tilde p \sim \rho_0 f_0 U L.
\end{equation}
From the vertical momentum equation at leading order, we also have:
\begin{equation}
  \pd{\tilde p}{z} \sim g \tilde \rho,
\end{equation}
which gives the density fluctuation scale:
\begin{equation}
  \tilde \rho \sim \frac{\rho_0 f_0 U L}{g H}.
\end{equation}
As a result we can write density and pressure as:
\begin{subequations}
  \begin{empheq}[left=\empheqlbrace]{alignat=2}
    p \,&=\ p_0(z) + \rho_0 f_0 U L \bar{p} = p_0(z) + \frac{\rho_0 U^2}{\text{Ro}} \bar p,\\[.5em]
    \rho \,&=\ \rho_0(z) \lp 1 + \frac{f_0 U L}{g H} \bar{\rho} \rp = \rho_0(z) \lp 1 + \frac{\text{Ro}}{\text{Bu}} \bar \rho \rp,
  \end{empheq}
\end{subequations}
Here, Ro is the Rossby number defined previously and $\text{Bu} =
  \nicefrac{L_D^2}{L^2}$ is the \textbf{Burger number}\index{Burger number}, with
$L_D = \sqrt{\nicefrac{g H}{f_0^2}}$ the \textbf{Rossby deformation
  radius}\index{Rossby deformation radius}. This characteristic lengthscale delineates flows which are dominated by buoyancy/gravity or geostrophy. Typically, the vortices developing in the Gulf Stream have a size comparable to $L_D \sim 30$ km.
For large (quasi-) geostrophic structures, $L \gg L_D$ and $\text{Bu} \ll 1$, but we still obtain a ratio Ro/Bu smaller than unity.

\subsection{Nondimensionalisation: a worked-out example}
Let's rewrite mass conservation with this new set of variables. Noting first
that the particle derivative may be written in terms of $x$, $y$ and $z$
derivatives as:
\begin{eqnarray}
  \dd{}{t} &&\equiv \pd{}{t} + \frac{u}{r\cos\theta} \pd{}{\phi} + \frac{v}{r} \pd{}{\theta} + w\pd{}{r}\notag\\[.5em]
  &&= \pd{}{t} + u \frac{R}{r} \frac{\cos \theta_0}{\cos \theta} \pd{}{x} + v \frac{R}{r} \pd{}{y} + w \pd{}{z},
\end{eqnarray}
and that the radius may be connected to the alitude as follows $r = R + H \bar z = R + \epsilon L \bar z = R \lp 1 + \epsilon \delta \bar z\rp$, with $\delta = \tfrac{L}{R}$, the mass conservation equation can be rewritten as:
\begin{multline}
  \rho_0 \frac{\text{Ro}}{\text{Bu}}\frac{U}{L} \dd{\bar \rho}{\bar t} + \rho_0 \lp 1 + \frac{\text{Ro}}{\text{Bu}}\bar \rho\rp \lp \frac{\epsilon U}{H} \frac{\bar w}{\rho_0} \pd{\rho_0}{\bar z} + 2 \frac{\epsilon U \bar w}{R \lp 1 + \epsilon \delta \bar z\rp} \right.\\
  \left. + \frac{\epsilon U}{H} \pd{\bar w}{\bar z} + \frac{R U}{R \lp 1 + \epsilon \delta \bar z \rp L} \pd{\bar v}{\bar y} - \frac{U}{R \lp 1 + \epsilon \delta \bar z \rp} \bar v \tan \theta + \frac{R \cos \theta_0}{R \lp 1 + \epsilon \delta \bar z \rp \cos \theta}\frac{U}{L}\pd{\bar u}{\bar x}\rp.
\end{multline}
This equation can be made dimensionless:
\begin{multline}
  \frac{\text{Ro}}{\text{Bu}} \dd{\bar \rho}{\bar t} + \lp 1 + \frac{\text{Ro}}{\text{Bu}}\bar \rho\rp \lp \frac{1}{\rho_0} \pd{\rho_0}{\bar z} \bar w + 2 \epsilon \delta \frac{\bar w}{1 + \epsilon \delta \bar z} + \pd{\bar w}{\bar z} + \frac{1}{1 + \epsilon \delta \bar z} \pd{\bar v}{\bar y} \right.\\
  \left. - \delta \frac{\bar v}{1 + \epsilon \delta \bar z} \tan \theta + \frac{\cos \theta_0}{\cos \theta} \frac{1}{1 + \epsilon \delta \bar z} \pd{\bar u}{\bar x}\rp = 0.
\end{multline}
So far no approximation has been made. But depending on the scale of the flow, several simplifications can be introduced.
\prg{Planetary-sized flows.} For flows at the planet scale, such as the polar vortex flow, we have Ro/Bu $\ll 1$, $\epsilon \ll 1$ and $\epsilon \delta \ll 1$.
Therefore, at leading order we obtain:
\begin{equation}
  \bar w \frac{1}{\rho_0} \pd{\rho_0}{\bar z} + \pd{\bar w}{\bar z} + \pd{\bar v}{\bar y} - \delta \bar v \tan \theta + \frac{\cos \theta_0}{\cos \theta}\pd{\bar u}{\bar x} = 0.
\end{equation}
In this context it is actually more relevant to express the equation back in dimensional form in spherical coordinates:
\begin{equation}
  \frac{\rho_0}{r \cos \theta}\lp \pd{u}{\phi} + \pd{\lp v \cos \theta\rp}{\theta} \rp + \pd{}{r} \lp \rho_0 w \rp = 0.
\end{equation}
\prg{Large-scale atmospheric/oceanic flows.} For large-scale atmospheric or oceanic flows, we still have Ro/Bu $\ll 1$ and $\epsilon \ll 1$, but now $\delta \ll 1$ as well.
At leading order we obtain the simpler equation in dimensional form:
\begin{equation}
  \pd{}{z} \lp \rho_0 w \rp + \rho_0 \lp \pd{u}{x} + \pd{v}{y} \rp = 0.
\end{equation}
\prg{Mesoscale flows.} For mesoscale flows, such as oceanic eddies, we have $\delta \ll 1$, $\epsilon \delta \ll 1$, Bu $\gg 1$, but neither Ro nor $\epsilon$ are small a priori.
At leading order we now obtain again:
\begin{equation}
  \pd{}{z} \lp \rho_0 w \rp + \rho_0 \lp \pd{u}{x} + \pd{v}{y} \rp = 0.
\end{equation}
\subsection{A hierarchy of models for planetary flows}
Applying the same scaling analysis to the momentum equations, we obtain a
hierarchy of models depending on the scale of the flow considered.
\prg{Planetary-sized flows.} At this scale, the radius of curvature of the
Earth cannot be neglected, and the flow appears to be hydrostatic at leading
order:
\begin{subequations}
  \begin{empheq}[left=\empheqlbrace]{alignat=2}
    \frac{\rho_0}{r \cos \theta}\lp \pd{u}{\phi} + \pd{\lp v \cos \theta\rp}{\theta} \rp + \pd{}{r} \lp \rho_0 w \rp & \, =\  0,\\
    \rho_0 \lp \dd{u}{t} - \frac{uv}{r} \tan \theta - 2 \Omega \sin \theta v \rp &\,=\ -\frac{1}{r \cos \theta} \pd{\tilde p}{\phi},\\
    \rho_0 \lp \dd{v}{t} + \frac{u^2}{r} \tan \theta + 2 \Omega \sin \theta u \rp &\,=\ -\frac{1}{r} \pd{\tilde p}{\theta},\\
    \pd{\tilde p}{r} &\,=\ -\tilde \rho g.
  \end{empheq}
\end{subequations}
\prg{Large-scale atmospheric/oceanic flows.} At this scale, the curvature of the Earth may be neglected, and the flow appears hydrostatic at leading order:
\begin{subequations}
  \begin{empheq}[left=\empheqlbrace]{alignat=2}
    \pd{u}{x} + \pd{v}{y} + \frac{1}{\rho_0} \pd{}{z} \lp \rho_0 w \rp & \,=\  0,\\
    \rho_0 \lp \dd{u}{t} - 2 \Omega \sin \theta_0 v \rp &\,=\ -\pd{\tilde p}{x},\\
    \rho_0 \lp \dd{v}{t} + 2 \Omega \sin \theta_0 u \rp &\,=\ -\pd{\tilde p}{y},\\
    \pd{\tilde p}{z} &\,=\ -\tilde \rho g.
  \end{empheq}
\end{subequations}
\prg{Mesoscale flows.} At this scale, the curvature of the Earth may be neglected, but the flow is not necessarily hydrostatic at leading order:
\begin{subequations}
  \begin{empheq}[left=\empheqlbrace]{alignat=2}
    &\rho_0\lp\pd{u}{x} + \pd{v}{y}\rp + \pd{}{z} \lp \rho_0 w \rp \,=\  0,\\
    &\rho_0 \dd{u}{t} \,=\ -\pd{\tilde p}{x} + f v - f' w,\\
    &\rho_0 \dd{v}{t} \,=\ -\pd{\tilde p}{y} - f u,\\
    &\rho_0 \dd{w}{t} \,=\ -\pd{\tilde p}{z} - \tilde \rho g + f' u.
  \end{empheq}
\end{subequations}
\section{Geostrophic equilibrium}\index{geostrophic equilibrium}
\label{sec:geostrophic_equilibrium}
Whenever nonlinear effects and viscous effects are negligible, planetary flows reach an equilibrium known as \textbf{geostrophic flow} where:
\begin{equation}
  \label{eq:geostrophic_balance}
  \rho 2 \bOmega \times \bu = -\nabla p.
\end{equation}
Using a local cartesian coordinate system with $\bi$ pointing eastwards, $\bj$ northwards and $\bk$ in the direction of the geopotential gradient, we can write:
\begin{equation}
  \label{eq:geostrophic_balance_cartesian}
  \bff \times \bu = -\frac{1}{\rho}\nabla p - \rho \bg_\text{eff}.
\end{equation}
with $\bff = f \bk$, $f = 2 \Omega \sin \theta$, and $\theta$ the latitude. Sometimes $f$ is approximated with a constant value $f_0$ ($f$-plane approximation) or linearized around a reference latitude $f = f_0 + \beta y$ ($\beta$-plane approximation).

In vertical projection, we recover the hydrostatic balance to first
approximation. In horizontal projection, we obtain:
\begin{subequations}
  \label{eq:geostrophic_balance_horizontal}
  \begin{empheq}[left=\empheqlbrace]{alignat=2}
    f u \,&=\ -\frac{1}{\rho}\pd{p}{y},\\
    f v \,&=\ \frac{1}{\rho}\pd{p}{x}.
  \end{empheq}
\end{subequations}
This implies that $f \bu \cdot \nabla p = 0$, which means that pressure is
constant along streamlines. The pressure plays the role of the streamfunction
and the wind follows the isobars! Further, if $f>0$ (i.e. in the northern
hemisphere), the wind is anticlockwise around a low pressure system (cyclonic
flows) and clockwise around a high pressure system (anticyclonic flows).
This is the \textbf{Buys-Ballot law}.

\section{Application: wind over France}
\label{sec:wind_france}

As an application we will now predict the wind over France using the
geostrophic approximation. We suppose that the wind follows the geostrophic
equilibrium and that pressure is hydrostatic, i.e.
\begin{equation}
  \pd{p}{z} = -\rho \pd{\phi}{z},
\end{equation}
where $\phi$ is the geopotential.

It is quite customary to use alternate variables for the vertical scale (it is
always possible as long as there is a bijection): pressure, entropy coordinates
are therefore routinely used. In pressure coordinates, we can make use of the
following. Let's take a variable $\psi$ depending on $(x, y, z, t)$ and let's
rewrite it as a function of $(x, y, \xi, t)$:
\begin{equation}
  \left.\pd{\psi}{\xi}\right|_{x,y,t} = \left.\pd{\psi}{z}\right|_{x,y,t} \left.\pd{z}{\xi}\right|_{x,y,t} \quad ; \quad \left.\pd{\psi}{z}\right|_{x,y,t} = \left.\pd{\psi}{\xi}\right|_{x,y,t} \left.\pd{\xi}{z}\right|_{x,y,t}.
\end{equation}
Similarly for the horizontal derivatives, we get:
\begin{equation}
  \left.\pd{\psi}{x}\right|_{\xi,y,t} = \left.\pd{\psi}{x}\right|_{z,y,t} + \left.\pd{\psi}{z}\right|_{x,y,t} \left.\pd{z}{x}\right|_{\xi,y,t}.
\end{equation}
So, in pressure coordinates:
\begin{equation}
  0 = \left.\pd{p}{x}\right|_{y,z,t}+ \left.\pd{p}{z}\right|_{x,y,t} \left.\pd{z}{x}\right|_{y,p,t}.
\end{equation}
and
\begin{equation}
  \left.\pd{p}{x}\right|_{y,z,t} = \rho \left.\pd{\phi}{x}\right|_{y,p,t}.
\end{equation}
and the geostrophic equilibrium reads:
\begin{equation}
  \label{eq:geostrophic_equilibrium_pressure}
  \bff \times \bu = -\frac{1}{\rho}\nabla_p \phi.
\end{equation}

\section{Oceanic and atmospheric circulation}
\label{sec:oceanic_atmospheric_circulation}

The fluid envelopes of the Earth are not static: they flow, destabilize into
fine turbulence and mix due to complex processes. But at larger scales, the
flows in the fluid shells exhibit remarkably stable patterns. Associated with
mass, momentum and especially energy transport these flows have a major impact
on the climate of the planet. In the following, we examine two remarkable
examples of such large-scale flows: the oceanic gyres and the emergence of
western boundary currents (such as the Gulf Stream) in the oceans, and the
Hadley cell in the atmosphere.
\subsection{Oceanic circulation: wind-induced gyres}
\label{sec:oceanic_circulation}

One of the most important factors influencing the climate is the sea surface
temperature, which exhibits variations along the latitude of course, but also
zonal variations (i.e. along the longitude). To make sense of this temperature
distribution, we need to understand the oceanic flows. These are structured in
large-scale \textbf{gyres}\index{gyres} (or vortices). It is noteworthy that
these gyres are asymmetric: the \textbf{western boundary
  currents}\index{western boundary currents} are much stronger, narrower and
deeper than the eastern boundary currents. In the North Atlantic, such an
intensification occurs along the US east coast: this is the famous \textbf{Gulf
  Stream}\index{Gulf Stream}, which transports warm water from the Caribbean up
to the North Atlantic, moderating the climate of Western Europe. In the
following we will see how these gyres form and why western boundary currents
are intensified.
\subsubsection{The key role of wind}
\label{sec:key_role_wind}
Somewhat counter-intuitively, the large-scale oceanic circulation is primarily
driven by the wind stress at the sea surface. Wind stress is best identified as
a source of oceanic motion by considering the African gyre in the Indian Ocean:
this gyre reverses seasonally, following the monsoon winds. This suggests that
wind is the main driver of the gyres, imparting a shear stress at the sea
surface and thereby setting the upper layer of the ocean in motion.
In the following, we model this process, following the pioneering work of Henry
Stommel \citep{Stommel1948}.
\subsubsection{Stommel's model for wind-driven gyres}
\label{sec:stommel_model}
We start with the equations expressing geostrophic equilibrium for the ocean:
\begin{equation}
  \label{eq:geostrophic_equilibrium_ocean}
  \rho \bff \times \bu = -\nabla p + \pd{}{z} \symbf{\tau},
\end{equation}
or in terms of components:
\begin{subequations}
  \begin{empheq}[left=\empheqlbrace]{alignat=2}
    -\rho fv \,&=\ -\pd{p}{x} + \pd{\tau_{xz}}{z},\\
    \rho fu \,&=\ - \pd{p}{y} + \pd{\tau_{yz}}{z}.
  \end{empheq}
  \label{eq:geostrophic_equilibrium_ocean_components}
\end{subequations}

\noindent Here, $\symbf{\tau} = \lp\tau_{xz}, \tau_{yz}, 0 \rp$ stands for the components of
the stress tensor associated with vertical transport. We model these
stresses using a turbulent viscosity $\nu_T$:
\begin{subequations}
  \begin{empheq}[left=\empheqlbrace]{alignat=2}
    \tau_{xz} \,&=\ \rho \nu_T \pd{u}{z},\\
    \tau_{yz} \,&=\ \rho \nu_T \pd{v}{z}.
  \end{empheq}
\end{subequations}
Taking the curl of the geostrophic equilibrium equations~\eqref{eq:geostrophic_equilibrium_ocean_components}, we obtain:
\begin{equation}
  \label{eq:curl_geostrophic_equilibrium_ocean}
  \pd{}{x} \lp \rho f u \rp + \pd{}{y} \lp \rho f v \rp = \pd{}{z} \lp \nabla \times \symbf{\tau} \rp.
\end{equation}
Introducing $\beta$, the northward gradient of the Coriolis parameter:
\begin{equation}
  \beta = \pd{f}{y},
\end{equation}
we can rewrite the equation~\eqref{eq:curl_geostrophic_equilibrium_ocean} as:
\begin{equation}
  \label{eq:curl_geostrophic_equilibrium_ocean_beta}
  \rho \lp \beta v + f \lp \pd{u}{x} + \pd{v}{y} \rp \rp = \pd{}{z} \lp \nabla \times \symbf{\tau} \rp.
\end{equation}
Using incompressibility, this equation becomes:
\begin{equation}
  \label{eq:curl_geostrophic_equilibrium_ocean_incompressible}
  \rho \beta v = \rho f \pd{w}{z} + \pd{}{z} \lp \nabla \times \symbf{\tau} \rp.
\end{equation}
Integrating this equation from the sea floor $z = z_b$ to the sea surface $z = z_t$ we obtain:
\begin{equation}
  \label{eq:curl_geostrophic_equilibrium_ocean_integrated}
  \rho \beta h \bar v = \rho f \left. w \right]_{z_b}^{z_t} + \left. \nabla \times \symbf{\tau} \right]_{z_b}^{z_t},
\end{equation}
This equation calls for a focus on the flow structure near the sea surface and sea floor,
where viscous and/or turbulent friction are predominant.
\subsubsection{Ekman layers and Ekman pumping}
\label{sec:ekman_layers}

Two completely equivalent choices can now be made: either the equation is
integrated exactly to the sea boundaries where the vertical velocity vanishes
(at the sea level the vertical velocity is zero on average), either the
integration is performed up to the edges of the Ekman layers. Within these
layers, the stress vary significantly to match the wind stress at the surface,
and the bottom friction on the sea floor, but outside these layers we suppose
that the horizontal components of the flow do not vary with height, i.e. that
$\symbf{\tau}$ is zero. The vertical velocity is however non-zero in the
vicinity of the Ekman layer -- a phenomenon known as \textbf{Ekman
  pumping}\index{Ekman pumping}.

The detailed analysis of the velocity field structure in the Ekman layers
reveals that within these regions of typical thickness $\delta = \sqrt{\nu_T /
    2 f}$, the horizontal velocity is observed to spiral and rotate by an angle of
$\pi/4$ with respect to the wind direction. This horizontal flow structure does
not satisfy $\pd{u}{x} + \pd{v}{y} = 0$ and is therefore associated with a
vertical velocity $w$. Near the sea surface we find:
\begin{equation}
  \label{eq:ekman_layer_surface}
  w = \frac{1}{\rho f} \nabla \times \symbf{\tau}_\text{wind}.
\end{equation}
At the sea floor, the vertical velocity can be expressed with the bottom friction or alternatively in terms of the geostrophic flow vorticity:
\begin{equation}
  \label{eq:ekman_layer_bottom}
  w = \frac{\delta}{2} \nabla \times \bu.
\end{equation}

\subsubsection{Stommel's equation}
Including these Ekman pumping effects in the integrated
equation~\eqref{eq:curl_geostrophic_equilibrium_ocean_integrated}, we obtain:
\begin{equation}
  \label{eq:stommel_equation}
  \rho \beta h \bar v = \nabla \times \symbf{\tau}_\text{wind} - \frac{1}{2} \rho f \delta \nabla \times \bar{\bu}.
\end{equation}
Introducing the streamfunction $\psi(x,y)$ such that $\bar{\bu} =-\nabla \times \lp \psi \be_z \rp$, we have:
\begin{equation}
  \bar u = -\pd{\psi}{y} \quad ; \quad \bar v = \pd{\psi}{x},
\end{equation}
we start by noting that $\nabla \times \bar{\bu} = \Delta \psi \be_z$.
We can therefore rewrite the previous equation as:
\begin{equation}
  \label{eq:stommel_equation_streamfunction}
  r \Delta \psi + \beta \pd{\psi}{x} = \frac{1}{\rho h}\nabla \times \symbf{\tau}_\text{wind},
\end{equation}
with $r = \frac{1}{2} f \delta / h$ a friction coefficient. This is \textbf{Stommel's
  equation}\index{Stommel's equation} for wind-driven oceanic gyres. On Earth, if the typical extent of an oceanic basin is $L \sim 5000$ km,
then we typically find that $\frac{r}{L} \ll \beta$ indicating that friction is weak compared to the $\beta$ effect.

\subsubsection{A boundary layer?}
As friction is small, it is tempting to neglect it altogether. Let's do so and
consider a model rectangular ocean basin of typical size $L$. On the basin
boundaries we impose $\psi = 0$. Denoting the typical wind stress scale as
$\tau_0$, we introduce the following non-dimensional variables:
\begin{subequations}
  \begin{empheq}[left=\empheqlbrace]{alignat=2}
    x, y \,&=\ L \bar{x}, L \bar{y},\\
    \tau \,&=\ \tau_0 \bar{\tau},\\
    \psi \,&=\ \frac{\tau_0}{\rho \beta h} \bar{\psi}.
  \end{empheq}
\end{subequations}
Noting $\epsilon = r / \beta L \ll 1$, Stommel's equation may be rewritten in non-dimensional form as:
\begin{equation}
  \label{eq:stommel_equation_nondim}
  \epsilon \Delta \bar{\psi} + \pd{\bar{\psi}}{\bar{x}} = \bar\nabla \times \bar{\tau}_\text{wind}.
\end{equation}
\prg{Outer solution.} Neglecting friction ($\epsilon = 0$) we obtain the outer solution:
\begin{equation}
  \label{eq:stommel_outer_solution}
  \bar{\psi}_0(\bar x, \bar y) = \int_0^x \bar\nabla \times \bar{\tau}_\text{wind} \,\mathrm d\xi + g(\bar y).
\end{equation}
associated with the velocity field:
\begin{subequations}
  \begin{empheq}[left=\empheqlbrace]{alignat=2}
    \bar{u}_0(\bar x, \bar y) \,&=\ -\pd{ }{\bar y}\int_0^x \bar\nabla \times \bar{\tau}_\text{wind} \,\mathrm d\xi - g'(\bar y),\\
    \bar{v}_0(\bar x, \bar y) \,&=\ \bar\nabla \times \bar{\tau}_\text{wind}.
  \end{empheq}
\end{subequations}
Let's now consider a model wind stress of the form:
\begin{equation}
  \bar{\tau}_\text{wind} = -\cos \lp \pi \bar{y} \rp \be_x,
\end{equation}
which corresponds to a wind blowing eastwards in the northern part of the basin
and westwards in the southern part of the basin. The outer solution is therefore:
\begin{equation}
  \bar{\psi}_0(\bar x, \bar y) = \pi \sin \lp \pi \bar{y} \rp \bar x + g(\bar y).
\end{equation}
The boundary conditions at $\bar y = 0$ and $\bar y = 1$ impose $g(0) = g(1) =
  0$. Imposing the western boundary condition corresponds to setting $g(\bar y) =
  0$. But imposing the eastern boundary condition leads to a contradiction with
$g(\bar y) = -\pi \sin(\pi \bar y)$. This means that the outer solution cannot
satisfy all boundary conditions: a boundary layer must form either at the
western or eastern boundary.
\prg{Boundary layer.} We now introduce a zoomed coordinate near the western or eastern boundary such
that $\bar x = \delta \tilde x$ if the boundary layer is on the west side (in
which case $\tilde x > 0$), and similarly $\bar x - 1 = \delta \tilde x$ if the
boundary layer is on the east side (in which case $\tilde x < 0$). In both
cases, Stommel's equation reads:
\begin{equation}
  \label{eq:stommel_equation_boundary_layer}
  \frac{\epsilon}{\delta^2} \pd{^2 \bar{\psi}}{\tilde x^2} + \pd{^2 \bar{\psi}}{\bar y^2} + \frac{1}{\delta}\pd{\bar{\psi}}{\tilde x} = \bar\nabla \times \bar{\tau}_\text{wind}.
\end{equation}
The distinguished scaling is obtained for $\delta = \epsilon$, leading to the boundary layer equation:
\begin{equation}
  \label{eq:stommel_equation_boundary_layer_distinguished}
  \pd{^2 \bar{\psi}}{\tilde x^2} + \pd{\bar{\psi}}{\tilde x} = 0.
\end{equation}
The solution for both boundary layers scenarios reads:
\begin{equation}
  \tilde{\psi} = A(\bar y) e^{-\tilde x} + B(\bar y) = A(\bar y) \lp e^{-\tilde x} - 1 \rp.
\end{equation}
While the western boundary layer ($\tilde x > 0$) decays exponentially away
from the boundary, the eastern boundary layer ($\tilde x < 0$) blows up
exponentially away from the boundary and is therefore not discarded. As a
result, the boundary layer must form on the western boundary, leading to an
intensified western boundary current, as observed in the Gulf Stream. Note that
the direction of the wind stress is of no importance here, as this forcing term
is absent from the boudnary layer equations.
\prg{Composite solution.} The composite solution for the streamfunction is therefore:
\begin{equation}
  \bar{\psi}(\bar x, \bar y) = \pi \sin(\pi \bar y) \lp \bar x - 1 \rp + A(\bar y) \lp e^{-\bar x / \epsilon} - 1 \rp + A(\bar y).
\end{equation}
The function $A(\bar y)$ is determined by imposing the matching condition:
\begin{equation}
  A(\bar y) = \pi \sin(\pi \bar y).
\end{equation}
As a result, the final solution reads:
\begin{equation}
  \bar{\psi}(\bar x, \bar y) = \pi \sin(\pi \bar y) \lp \bar x - 1 + e^{-\bar x / \epsilon} \rp.
\end{equation}
This solution exhibits a strong western boundary current of typical width
$\epsilon L = r / \beta$: the width of the western boundary current is
therefore controlled by the friction coefficient $r$ and the $\beta$ effect.