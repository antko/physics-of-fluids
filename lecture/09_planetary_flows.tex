% !TEX root = ./physics_of_fluids.tex
% !TEX TS-program = xelatex
% !TEX encoding = UTF-8 Unicode
\chapter[Planetary flows and the effect of rotation]{{\bfseries Planetary} flows and the effect of {\bfseries rotation}} \label{chap:planetary_flows}

Flows at the Earth scale are characterized by considerable horizontal
lengthscales (scaling with the radius of the Earth $\sim$ 6000 km) and
relatively small vertical lengthscales -- atmosphere is typically 10 km-thick
and oceans 4 km-deep. The ratio between these lengthscales is $O(1000)$ (which
remarkably is similar to the ratio between the radius of a soap bubble and the
thickness of the liquid film!). This contrast between lengthscales implies that
flows at the Earth-scale will be well captured by thin-film equations.

Another key aspect of planetary flows is \textbf{rotation}, which gives rise to
particular dynamics, of which cyclones are probably the most impressive
illustration. Cyclones owe their existence to the Coriolis force, which
imprints its mark on motion over large scales on Earth.

\section{Rotating fluids}
\label{sec:rotating_fluids}
The equations for motion in a rotating frame are best obtained using the Lagrangian formalism. At the fluid particle level, Newton's equation for motion is written:
\begin{equation}
  \label{eq:newton}
  \rho \dd{^2\br}{t^2} = \bff
\end{equation}
in an inertial frame. This equation can be rewritten in a rotating frame as:
\begin{equation}
  \label{eq:newton_rot}
  \rho \lp\dd{^2\br}{t^2} + 2 \bOmega \times \dd{\br}{t} + \bOmega \times \lp\bOmega \times \br\rp\rp = \bff,
\end{equation}
or:
\begin{equation}
  \label{eq:newton_rot_euler}
  \rho \lp\dd{\bu}{t} + 2 \bOmega \times \bu + \bOmega \times \lp\bOmega \times \br\rp\rp = \bff.
\end{equation}
Expressing the forces, we obtain:
\begin{equation}
  \label{eq:newton_rot_euler_forces}
  \rho \lp\dd{\bu}{t} + 2 \bOmega \times \bu + \bOmega \times \lp\bOmega \times \br\rp\rp = -\nabla p + \mu \Delta \bu + \rho \bg_\text{grav}.
\end{equation}
The centrifuge force part can be absorbed in a potential called the \textbf{geopotential} $\phi$:
\begin{equation}
  \label{eq:geopotential}
  \phi = g_\text{grav}z + \frac{1}{2} \lp\bOmega \times \br\rp^2.
\end{equation}
Introducing the effective gravity $\bg_\text{eff} = -\nabla \phi = \bg_\text{grav} - \bOmega \times \lp\bOmega \times \br\rp$, we can rewrite the equation of motion as:
\begin{equation}
  \label{eq:newton_rot_euler_forces_eff}
  \rho \lp\dd{\bu}{t} + 2 \bOmega \times \bu\rp = -\nabla p + \mu \Delta \bu - \rho \nabla \phi.
\end{equation}

We can non-dimensionalize this equation:
\begin{equation}
  \label{eq:newton_rot_euler_forces_eff_nondim}
  \frac{\rho U}{T} \pd{\bar{\bu}}{\bar t} + \frac{\rho U^2}{L} \lp \bar{\bu} \cdot \bar \nabla\rp \bar{\bu} + 2 \Omega \rho U \be_z \times \bar{\bu} = -\frac{P}{L} \bar\nabla \bar p + \frac{\mu U}{L^2} \bar\Delta \bar{\bu} - \frac{\rho \Phi}{L} \bar\nabla \bar\phi.
\end{equation}
In this expression, we will take for the natural time scale of the problem $T = \lp 2 \Omega\rp^{-1}$ which is the Coriolis pulsation. Further posing $P = \rho U \lp 2 \Omega \rp L$ and $\Phi = 2 \Omega U$ to keep the corresponding terms in the equation, we obtain:
\begin{equation}
  \label{eq:newton_rot_euler_forces_eff_nondim2}
  \pd{\bar{\bu}}{\bar t} + \text{Ro} \lp \bar{\bu} \cdot \bar \nabla\rp \bar{\bu} + \be_z \times \bar{\bu} = -\bar\nabla \bar p + E \bar\Delta \bar{\bu} - \bar\nabla \bar\phi.
\end{equation}
Two non-dimensional parameters appear in this equation:
\begin{itemize}
  \item the \textbf{Rossby number} $\text{Ro} = \frac{U}{2 \Omega L}$, which quantifies
        the relative importance of inertial terms with respect to the Coriolis force,
        and
  \item the \textbf{Ekman number} $E = \frac{\mu}{2 \rho \Omega L^2}$, which measures
        viscous friction relative to Coriolis action.
\end{itemize}
With a rotation rate of $\Omega = 7.3 \times 10^{-5} \text{s}^{-1}$,
we see that atmospheric winds of $\sim$100 km/h (28 m/s) will be dominated by rotation if $L \gtrsim 200$ km.
Oceanic flows are much slower; the Gulf Stream, which flows at $U \sim 1 \text{ m/s}$ sees this scale fall down to $7$ km. For the same oceanic flows, the Ekman number is $E \sim 10^{-10}$, showing that viscous friction is negligible compared to Coriolis action.

Flows at the planetary scale are therefore strongly influenced by rotation, and
to first approximation these flows result from a balance between pressure
gradient and Coriolis force. This is the \textbf{geostrophic balance}.

\section{Fluid flows on a rotating sphere}
\label{sec:flows_rotating_sphere}
The flows developing on the planet are complex but, depending on their scales,
simplifications may be made. Let's investigate these different regimes,
starting by writing the equations of motion on a rotating sphere.

\subsection{Governing equations in spherical coordinates}
We have for mass conservation:
\begin{equation}
  \label{eq:mass conservation_sphere}
  \frac{\mathrm d\rho}{\mathrm dt} + \rho \left(\pd{w}{r} + \frac{2w}{r} + \frac{1}{r \cos \theta} \pd{\lp v \cos \theta \rp}{\theta} + \frac{1}{r \cos \theta} \pd{u}{\phi}\right) = 0,
\end{equation}
where $(u, v, w)$ are the velocity components in the $(\phi, \theta, r)$
directions respectively (i.e. longitude, latitude, radius). $\dd{}{t}$ stands for the particle derivative operator\index{derivative!particle}, which reads in spherical coordinates:
\begin{equation}
  \label{eq:particle_derivative_spherical}
  \dd{}{t} = \pd{}{t} + \frac{u}{r \cos \theta} \pd{}{\phi} + \frac{v}{r} \pd{}{\theta} + w \pd{}{r}.
\end{equation}
The momentum equations take the following form:
\begin{subequations}
  \begin{empheq}[left=\empheqlbrace]{alignat=2}
    \rho \lp \dd{u}{t} + \frac{uw}{r} - \frac{uv}{r} \tan \theta - 2 \Omega \sin \theta v + 2 \Omega \cos \theta w\rp \,&=\ -\frac{1}{r \cos \theta} \pd{p}{\phi} + \matF_\phi,\\
    \rho \lp \dd{v}{t} + \frac{vw}{r} + \frac{u^2}{r} \tan \theta + 2 \Omega \sin \theta u \rp \,&=\ -\frac{1}{r} \pd{p}{\theta} + \matF_\theta,\\
    \rho \lp \dd{w}{t} - \frac{u^2+v^2}{r} - 2 \Omega \cos \theta u \rp \,&=\ -\pd{p}{r} - \rho g + \matF_r.
  \end{empheq}
\end{subequations}
Finally the energy equation reads:
\begin{equation}
  \label{eq:energy_sphere}
  \rho c_p \dd{T}{t} = k \Delta T + Q.
\end{equation}
Here $Q$ represents a possible volumetric heating (e.g. radiative heating from the Sun).

Not all the terms in these equations are equally important. In order to
identify the leading terms in the previous equations we \textbf{choose}
relevant scales for the problem and non-dimensionalize the equations.

\subsection{A local cartesian coordinate system}
\label{sec:local_cartesian}
In order to simplify the previous system of equations, we define the following variables:
\begin{subequations}
  \begin{empheq}[left=\empheqlbrace]{alignat=2}
    x \,&=\ R \phi \cos \theta_0,\\
    y \,&=\ R (\theta - \theta_0),\\
    z \,&=\ r - R,
  \end{empheq}
\end{subequations}
and relate the derivatives in the two coordinate systems:
\begin{subequations}
  \begin{empheq}[left=\empheqlbrace]{alignat=2}
    \pd{}{\phi} \,&=\ R \cos \theta_0 \pd{}{x},\\
    \pd{}{\theta} \,&=\ R \pd{}{y},\\
    \pd{}{r} \,&=\ \pd{}{z}.
  \end{empheq}
\end{subequations}
\subsection{Scaling the equations}
We now introduce characteristic scales and non-dimensional variables:
\begin{subequations}
  \begin{empheq}[left=\empheqlbrace]{alignat=2}
    x, y \,&=\ L \bar{x}, L \bar{y},\\
    z \,&=\ H \bar{z},\\
    u, v \,&=\ U \bar{u}, U \bar{v},\\
    w \,&=\ \epsilon \bar{w},\\
    t \,&=\ \frac{L}{U} \bar{t},
  \end{empheq}
\end{subequations}
with $\epsilon = H/L \ll 1$ the aspect ratio of the flow.

In order to scale pressure and density, we will consider the the
atmospheric/oceanic flows to be fluctuations around the hydrostatic equilibrium
given by
\begin{equation}
  \pd{p_0}{z} = -\rho_0 g,
\end{equation}
i.e. we write:
\begin{subequations}
  \begin{empheq}[left=\empheqlbrace]{alignat=2}
    p \,&=\ p_0(z) + \tilde{p},\\
    \rho \,&=\ \rho_0(z) + \tilde{\rho}.
  \end{empheq}
\end{subequations}
If we are located mid-latitude, we expect the pressure to be balanced by the Coriolis acceleration:
\begin{equation}
  \rho_0 2 \Omega \sin \theta_0 U \sim \frac{\tilde p}{L}.
\end{equation}
Noting $f_0 = 2 \Omega \sin \theta_0$ the Coriolis parameter at latitude $\theta_0$, we can therefore scale pressure fluctuations as:
\begin{equation}
  \tilde p \sim \rho_0 f_0 U L.
\end{equation}
From the vertical momentum equation at leading order, we also have:
\begin{equation}
  \pd{\tilde p}{z} \sim g \tilde \rho,
\end{equation}
which gives the density fluctuation scale:
\begin{equation}
  \tilde \rho \sim \frac{\rho_0 f_0 U L}{g H}.
\end{equation}
As a result we can write density and pressure as:
\begin{subequations}
  \begin{empheq}[left=\empheqlbrace]{alignat=2}
    p \,&=\ p_0(z) + \rho_0 f_0 U L \bar{p} = p_0(z) + \frac{\rho_0 U^2}{\text{Ro}} \bar p,\\[.5em]
    \rho \,&=\ \rho_0(z) \lp 1 + \frac{f_0 U L}{g H} \bar{\rho} \rp = \rho_0(z) \lp 1 + \frac{\text{Ro}}{\text{Bu}} \bar \rho \rp,
  \end{empheq}
\end{subequations}
Here, Ro is the Rossby number defined previously and $\text{Bu} =
  \nicefrac{L_D^2}{L^2}$ is the \textbf{Burger number}\index{Burger number}, with
$L_D = \sqrt{\nicefrac{g H}{f_0^2}}$ the \textbf{Rossby deformation
  radius}\index{Rossby deformation radius}. This characteristic lengthscale delineates flows which are dominated by buoyancy/gravity or geostrophy. Typically, the vortices developing in the Gulf Stream have a size comparable to $L_D \sim 30$ km.
For large (quasi-)geostrophic structures, $L \gg L_D$ and $\text{Bu} \ll 1$, but we still obtain a ratio Ro/Bu smaller than unity.

\section{Geostrophic flows}\index{geostrophic flows}
\label{sec:geostrophic_flows}
Whenever nonlinear effects and viscous effects are negligible, planetary flows reach an equilibrium known as \textbf{geostrophic flow} where:
\begin{equation}
  \label{eq:geostrophic_balance}
  \rho 2 \bOmega \times \bu = -\nabla p.
\end{equation}
Using a local cartesian coordinate system with $\bi$ pointing eastwards, $\bj$ northwards and $\bk$ in the direction of the geopotential gradient, we can write:
\begin{equation}
  \label{eq:geostrophic_balance_cartesian}
  \bff \times \bu = -\frac{1}{\rho}\nabla p - \rho \bg_\text{eff}.
\end{equation}
with $\bff = f \bk$, $f = 2 \Omega \sin \theta$, and $\theta$ the latitude. Sometimes $f$ is approximated with a constant value $f_0$ ($f$-plane approximation) or linearized around a reference latitude $f = f_0 + \beta y$ ($\beta$-plane approximation).

In vertical projection, we recover the hydrostatic balance to first
approximation. In horizontal projection, we obtain:
\begin{subequations}
  \label{eq:geostrophic_balance_horizontal}
  \begin{empheq}[left=\empheqlbrace]{alignat=2}
    f u \,&=\ -\frac{1}{\rho}\pd{p}{y},\\
    f v \,&=\ \frac{1}{\rho}\pd{p}{x}.
  \end{empheq}
\end{subequations}
This implies that $f \bu \cdot \nabla p = 0$, which means that pressure is constant along streamlines. The pressure plays the role of the streamfunction and the wind follows the isobars! Further, if $f>0$ (i.e. in the northern hemisphere), the wind is clockwise around a low pressure system (cyclonic flows) and anticlockwise around a high pressure system (anticyclonic flows). This is the \textbf{Buys-Ballot law}.

\section{Application: wind over France}
\label{sec:wind_france}

As an application we will now predict the wind over France using the
geostrophic approximation. We suppose that the wind follows the geostrophic
equilibrium and that pressure is hydrostatic, i.e.
\begin{equation}
  \pd{p}{z} = -\rho \pd{\phi}{z},
\end{equation}
where $\phi$ is the geopotential.

It is quite customary to use alternate variables for the vertical scale (it is
always possible as long as there is a bijection): pressure, entropy coordinates
are therefore routinely used. In pressure coordinates, we can make use of the
following. Let's take a variable $\psi$ depending on $(x, y, z, t)$ and let's
rewrite it as a function of $(x, y, \xi, t)$:
\begin{equation}
  \left.\pd{\psi}{\xi}\right|_{x,y,t} = \left.\pd{\psi}{z}\right|_{x,y,t} \left.\pd{z}{\xi}\right|_{x,y,t} \quad ; \quad \left.\pd{\psi}{z}\right|_{x,y,t} = \left.\pd{\psi}{\xi}\right|_{x,y,t} \left.\pd{\xi}{z}\right|_{x,y,t}.
\end{equation}
Similarly for the horizontal derivatives, we get:
\begin{equation}
  \left.\pd{\psi}{x}\right|_{\xi,y,t} = \left.\pd{\psi}{x}\right|_{z,y,t} + \left.\pd{\psi}{z}\right|_{x,y,t} \left.\pd{z}{x}\right|_{\xi,y,t}.
\end{equation}
So, in pressure coordinates:
\begin{equation}
  0 = \left.\pd{p}{x}\right|_{y,z,t}+ \left.\pd{p}{z}\right|_{x,y,t} \left.\pd{z}{x}\right|_{y,p,t}.
\end{equation}
and
\begin{equation}
  \left.\pd{p}{x}\right|_{y,z,t} = \rho \left.\pd{\phi}{x}\right|_{y,p,t}.
\end{equation}
and the geostrophic equilibrium reads:
\begin{equation}
  \label{eq:geostrophic_equilibrium_pressure}
  \bff \times \bu = -\frac{1}{\rho}\nabla_p \phi.
\end{equation}